\part{Introducción}
% Aqui debemos colocar una introducción para comentar las pruebas y sucesos que se van a llevar a cabo
La mejora de contraste es una operación fundamental en el procesamiento de imágenes, ampliamente utilizada en diversos campos científicos y tecnológicos.

Entre las técnicas más utilizadas, la ecualización de histograma destaca por su efectividad y simplicidad. A través del análisis y transformación del histograma, es posible identificar y corregir regiones con intensidades predominantes, obteniendo un resultado visual más equilibrado y de mayor calidad.

El presente trabajo tiene como objetivo optimizar una aplicación secuencial para la mejora de contraste de imágenes mediante la ecualización del histograma, aprovechando las capacidades de paralelización ofrecidas por MPI y OpenMP.

El documento presenta un análisis del programa secuencial, una implementación\newline OpenMP, MPI y un formato híbrido que aproveche ambos \textit{frameworks}.

En cada sección, se incluye una descripción metodológica basada en las cuatro fases de paralelización: descomposición, asignación, orquestación y reparto. Además, se presentan gráficos que muestran los tiempos de ejecución y la aceleración lograda en comparación con la versión secuencial. 

Finalmente, las conclusiones recogen un análisis comparativo del rendimiento de las distintas estrategias de paralelización, destacando sus ventajas, limitaciones y casos de uso óptimos. Este trabajo tiene como meta no solo mejorar el tiempo de ejecución en el procesamiento del contraste de imágenes, sino también explorar y comprender cómo los paradigmas de paralelismo pueden ser utilizados para resolver problemas computacionalmente intensivos de manera eficiente.