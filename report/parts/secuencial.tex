\section{Programa secuencial}
% Aquí debemos incluir los tiempos del código secuencial, de primera mano, quizás con gráficas añadidas al igual que como vamos a hacer después con el código paralelo a modo de introducción
El programa base, presentado en el enunciado de la práctica, ofrece una ecualización de histograma, así como sus respectivos pasos para calcularlo.

Está presente un reparto homogéneo de iteraciones y cálculos redundantes, con gran cantidad de accesos y vectores. La \figref{hslsecuencial}, \figref{yuvsecuencial} y \figref{grisessecuencial} recogen las métricas de ejecución obtenidas en las máquinas del laboratorio, Avignon.
Será turno de los desarrolladores y de los \textit{frameworks} OpenMP y MPI reducir los tiempos, además de mejorar la eficiencia y eficacia al programa.

\svgfigure[1]{hslsecuencial}{Tiempo de ejecución HSL del programa secuencial.}
\newpage
\svgfigure[1]{yuvsecuencial}{Tiempo de ejecución YUV del programa secuencial.}
\svgfigure[1]{grisessecuencial}{Tiempo de ejecución del procesado de grises del programa secuencial.}