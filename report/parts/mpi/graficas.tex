\subsection{Métricas}
% Justificar y adjuntar las gráficas de MPI, explicando resultados y gráficas
En esta sección, se incluyen las métricas obtenidas durante la ejecución del programa con paralelismo \textbf{MPI}. Quedan recogidas representaciones gráficas, agrupados dos a dos (HSL y YUV para color y grises), un tipo de representación expone una comparativa con la ejecución secuencial en términos de tiempo, otra en términos de aceleración y una tercera recoge una equiparación del \textit{speed-up} empírico versus la \textbf{ley de Gustafson}. 
Se dispone de un análisis posterior de las mismas, con el objetivo de definir conclusiones que demuestren qué tan buen rendimiento puede llegar a obtener MPI con la optimización implementada.

Las fórmulas utilizadas para calcular el speedup y la ley Gustafson son las mismas definidas en la sección de \ref{Métricas de OpenMP}{Métricas de OpenMP}.

\svgfigure[0.8]{seq vs mpi - HSL}{Comparativa de tiempos de ejecución HSL.}
\svgfigure[0.8]{seq vs mpi - HSL speedup}{Aceleración empírica del programa MPI procesando HSL respecto al programa secuencial.\textsuperscript{\ref{note:speedup}}}

\svgfigure[1]{seq vs mpi - YUV}{Comparativa de tiempos de ejecución YUV.}
\svgfigure[1]{seq vs mpi - YUV speedup}{Aceleración empírica del programa MPI procesando YUV respecto al programa secuencial.}

\newpage
\svgfigure[1]{seq vs mpi - grey}{Comparativa de tiempos de ejecución de la escala de grises.}
\svgfigure[1]{seq vs mpi - grey speedup}{Aceleración empírica del programa MPI procesando la escala de grises respecto al programa secuencial.}

Como puede apreciarse, las métricas y visualizaciones anteriores muestran un patrón consistente para el procesamiento de color y escala de grises. 
La paralelización con MPI resulta en mejoras significativas en el rendimiento, especialmente en los primeros incrementos de hilos, pero el impacto disminuye gradualmente debido a factores inherentes al paralelismo como \textit{overheads}, sincronización y porciones no paralelizables del código. 
Este mismo overhead causa que el resultado obtenido usando más de un nodo en la escala de grises resulte en una \textbf{pérdida} de eficiencia (speedup inferior a 1) para más de un (1) nodo.
La aceleración o \textit{speed-up} conseguido, ya sea mediante el uso de más procesos en escala de grises o mediante el uso de más nodos y procesos, demuestra la posibilidad de mejorar eficiencia a la hora de resolver el problema, hasta límites prácticos impuestos por la arquitectura y la naturaleza del problema.

\newpage
A continuación, se recoge una comparación de la aceleración empírica respecto a la \textbf{ley de Gustafson} para cada caso anterior, calculando la alfa para cada conjunto y a continuación mostrando el speedup acorde a ambas medidas.
En general, la \figref{seq vs mpi - YUV speedup gustafson}, la \figref{seq vs mpi - HSL speedup gustafson} y la \figref{seq vs mpi - grey speedup gustafson} muestran un patrón común: la aceleración teórica supera a la empírica, lo que refleja las limitaciones del paralelismo en un entorno práctico. 
Los tres casos sugieren que el sistema experimenta una saturación o una reducción en el paralelismo alrededor de los 4-8 procesos, potencialmente debido a problemas de sincronización o una carga de trabajo mal distribuida. 
Los problemas de sincronización son uno de los principales problemas a la hora de paralelizar, puesto que el tiempo usado en comunicarse y sincronizar limita el tiempo máximo paralelizable.

Cómo se había comentado previamente, el principal caso llamativo es la escala de grises, ya que puesto a que pierde eficiencia respecto al modelo secuencial, la ley de gustafson prevee una pérdida de eficiencia mayor que el estancamiento real. 
Esto resalta la importancia de evaluar la eficiencia tras paralelizar, para compribar que la implementación realizada sea de verdad una mejora. 
Resalta también como no todas las tareas mejoran mediante paralelización.

\svgfigure[.8]{seq vs mpi - YUV speedup gustafson}{Aceleración empírica del programa MPI procesando YUV respecto al cálculo teórico de Gustafson.}
\svgfigure[.8]{seq vs mpi - HSL speedup gustafson}{Aceleración empírica del programa MPI procesando HSL respecto al cálculo teórico de Gustafson.}
\svgfigure[.8]{seq vs mpi - grey speedup gustafson}{Aceleración empírica del programa MPI procesando escala de grises respecto al cálculo teórico de Gustafson.}