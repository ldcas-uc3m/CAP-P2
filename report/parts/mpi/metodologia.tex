\subsection{Metodología}
% Incluir la metodología seguida para cada caso cambiado utilizando los steps del enunciado: descomposición, asignación, orquestación y reparto

Como ya hemos mencionado anteriormente, la metodología de paralelización se puede abstraer en cuatro fases principales.

En MPI, se generan un número de procesos repartidos en una serie de nodos.
Para la implementación con MPI, en la fase de \textbf{descomposición} se divide la imagen de entrada (de ancho $w$ y altura $h$) entre el número total de procesos $n$\footnote{El primer proceso será el proceso $p_0$ y el último será el proceso $p_{n-1}$. El proceso $p_0$ será considerado también como proceso \textit{root}, o proceso orquestador.}. Éste proceso se realiza por filas, dado que el lenguaje de programación, C++, es \textit{row major}. Para cubrir el caso de que la imagen no se pueda dividir equitativamente en filas, el resto de filas $r = h \text{ mod } w$ se repartirán entre los $r$ primeros procesos.

Para definir la \textbf{asignación} los elementos de cada \textit{chunk} usaremos el número de elementos del mismo, $C(p_i)$, y el desplazamiento, $D(p_i)$, entendido como el índice del primer elemento de la imagen que pertenece al mismo.
Estos valores quedan definidos por las ecuaciones \ref{eq:count} y \ref{eq:displacement}.


\begin{equation}\label{eq:count}
  C(p_i) =
  \begin{cases}
    \lfloor h/w\rfloor + rw ,& \text{si } i = 0\\
    \lfloor h/w\rfloor + rw,& \text{si } 0 < i \le r\\
    \lfloor h/w\rfloor,& \text{si } i > r\
  \end{cases}
\end{equation}

\begin{equation}\label{eq:displacement}
  D(p_i) =
  \begin{cases}
    0,& \text{if } i = 0\\
    D(p_{i-1}) + C(p_{i-1}) - w,& \text{si } i > 0
  \end{cases}
\end{equation}


Dado que para calcular la ecualización del histograma de un píxel (elemento) es necesario conocer los valores de los píxeles que le rodean, a cada trozo (o \textit{chunk}) que se le asigne a cada proceso $p_i$ se le añadirán los elementos de la fila inmediatamente superior e inferior. En el caso de que no exista esa fila (el caso del primer y el último \textit{chunk}), esto no será necesario. Ésto genera un solapamiento que será obviado a la hora de generar la imagen final. El reparto inicial, por tanto, queda definido por la ecualización \ref{eq:count_}.

\begin{equation}\label{eq:count_}
  C'(p_i) =
  \begin{cases}
    \lfloor h/w\rfloor + (1+r) \cdot w ,& \text{si } i = 0\\
    \lfloor h/w\rfloor + (2+r) \cdot w,& \text{si } 0 < i \le r\\
    \lfloor h/w\rfloor + 2w,& \text{si } r < i < n - 1\\
    \lfloor h/w\rfloor + w,& \text{si } i = n - 1\\
  \end{cases}
\end{equation}



Para la \textbf{orquestación} y el \textbf{reparto} de los elementos se realizarán llamadas MPI. El proceso seguido para cada uno de los sub-procesos (escala de grises, color HSL y color YUV), es el siguiente:
\begin{enumerate}
  \item El proceso $p_0$ lee el fichero de entrada, y envía, mediante una llamada \texttt{MPI\_Bcast} el tamaño de la imagen al resto de procesos.
  \item Todos los procesos calculan los distintos tamaños y desplazamientos iniciales ($C'(p)$ y $D(p$)).
  \item Se realiza una llamada \texttt{MPI\_Scatterv} por canal para que el proceso $p_0$ envíe los datos correspondientes al resto de procesos, generando una sub-imagen en cada proceso.\footnote{Para el caso de las imágenes a color, en este punto también se transforma la sub-imagen de RGB a HSL/YUV.}
  \item Cada proceso calcula su histograma parcial. En éste histograma no se tienen en cuenta las filas extras de solapamiento ($C(p)$ y $D(p$)).
  \item Se realiza una llamada \texttt{MPI\_Allreduce} para generar el histograma de la imagen completa y repartirlo a todos los nodos.
  \item Cada proceso realiza la ecualización de su sub-imagen. Es importante recalcar que para este cálculo se tiene en cuenta el histograma de la imagen completa.
  \item Los procesos vuelven a computar los desplazamientos y tamaños, excluyendo en esta ocasión las filas extras de solapamiento ($C(p)$ y $D(p$)).
  \item Se realiza una llamada \texttt{MPI\_Gatherv} por canal para generar la imagen final en el proceso $p_0$.
  \item El proceso $p_0$ escribe el fichero de salida.
\end{enumerate}