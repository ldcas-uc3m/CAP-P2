\subsection{Métricas}
En esta sección, se incluyen las métricas obtenidas durante la ejecución del programa con paralelismo \textbf{OpenMP}. Quedan recogidas representaciones gráficas, agrupados dos a dos (HSL y YUV para color y grises), un tipo de representación expone una comparativa con la ejecución secuencial en términos de tiempo, otra en términos de aceleración y una tercera recoge una equiparación del \textit{speed-up} empírico versus la \textbf{ley de Gustafson}. Se dispone de un análisis posterior de las mismas, con el objetivo de definir conclusiones que demuestren qué tan buen rendimiento puede llegar a obtener OpenMP con la optimización implementada.
\rasterfigure[0.9]{hslopenmp1}{Comparativa de tiempos de ejecución HSL.}
\rasterfigure[0.9]{hslopenmp2}{Aceleración empírica del programa OpenMP procesando HSL respecto al programa secuencial.\textsuperscript{\ref{note:speedup}}}
\footnote{\label{note:speedup}La fórmula utilizada para medir el \textit{speed-up} experimental es 
$\text{Speed-up} = \frac{T_{s_n}}{T_{p_n}}$, donde $T_{s_n}$ es el tiempo secuencial medido con $n$ procesos y $T_{p_n}$ es el tiempo medido del programa paralelo con OpenMP utilizando $n$ hilos. Esto genera un margen de error asumible que podría ser solventado con la Ley de Amdahl.}

\rasterfigure[1]{yuvopenmp1}{Comparativa de tiempos de ejecución YUV.}
\rasterfigure[1]{yuvopenmp2}{Aceleración empírica del programa OpenMP procesando YUV respecto al programa secuencial.}

\newpage
\rasterfigure[1]{grisesopenmp1}{Comparativa de tiempos de ejecución de la escala de grises.}
\rasterfigure[1]{grisesopenmp2}{Aceleración empírica del programa OpenMP procesando la escala de grises respecto al programa secuencial.}

Como puede apreciarse, las métricas y visualizaciones anteriores muestran un patrón consistente para el procesamiento de color y escala de grises. La paralelización con OpenMP resulta en mejoras significativas en el rendimiento, especialmente en los primeros incrementos de hilos, pero el impacto disminuye gradualmente debido a factores inherentes al paralelismo como \textit{overheads}, sincronización y porciones no paralelizables del código. La aceleración o \textit{speed-up} conseguido demuestra un uso eficiente de recursos hasta límites prácticos impuestos por la arquitectura y la naturaleza del problema.
\newpage
A continuación, se recoge una comparación de la aceleración empírica respecto a la \textbf{ley de Gustafson} para cada caso anterior, expuesta en la siguiente fórmula:
\[
S(N) = N - \alpha \cdot (N - 1)
\]

El factor \textit{alpha} queda indicado como la fracción del código que es secuencial, y por ende, no paralelizable. En este caso, queda resumido en la siguiente fórmula su cálculo. Se obtiene despejando en la \textbf{ley de Gustafson} y utilizando los datos de la aceleración empírica:
\[
\alpha = \frac{N - S(N)}{N - 1}
\]

Posteriormente, realizando la media de las fracciones obtenidas por N procesos, se define un valor para cada caso:

\rasterfigure[.3]{alphaopenmp}{Fracciones obtenidas para cada caso.}

Finalmente, se incluyen las distintas visualizaciones mencionadas. En general, la \figref{yuvopenmp3}, la \figref{hslopenmp3} y la \figref{grisesopenmp3} muestran respectivamente un patrón común: la aceleración teórica siempre supera a la empírica, lo que refleja las limitaciones del paralelismo en un entorno práctico. Los tres casos sugieren que el sistema experimenta una saturación o una reducción en el paralelismo, posiblemente debido a problemas de sincronización o una carga de trabajo mal distribuida. 

Entrando más en detalle, la discrepancia más notoria se encuentra en el procesado YUV, que enfrenta mayores desafíos de escalabilidad y complejidad, mientras que HSL logra un mejor rendimiento en configuraciones con mayor número de procesos. Esto resalta la importancia de optimizar tanto el software como el hardware para minimizar estos factores y cerrar la brecha entre la teoría y la práctica.

\rasterfigure[.8]{yuvopenmp3}{Aceleración empírica del programa OpenMP procesando YUV respecto al cálculo teórico de Gustafson.}
\rasterfigure[.8]{hslopenmp3}{Aceleración empírica del programa OpenMP procesando YUV respecto al cálculo teórico de Gustafson.}
\rasterfigure[.8]{grisesopenmp3}{Aceleración empírica del programa OpenMP procesando YUV respecto al cálculo teórico de Gustafson.}