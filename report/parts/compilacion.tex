\section{Compilación y ejecución}


\subsection*{Compilación}
Para compilar este programa es necesario tener instalado en el sistema las siguientes dependencias:
\begin{itemize}
  \item OpenMP\footurl{https://www.openmp.org/}
  \item MPICH\footurl{https://www.mpich.org/}
  \item CMake\footurl{https://cmake.org/}
  \item GNU Make\footurl{https://www.gnu.org/software/make/}, o similar
\end{itemize}


\noindent
Es necesario ejecutar los siguientes comandos desde la raíz del repositorio para realizar la compilación:
\begin{verbatim}
  mkdir build
  cd build/
  cmake .. -DCMAKE_CXX_COMPILER=mpicxx.mpich
  make
\end{verbatim}

\noindent
Ésto generará cuatro ejecutables, correspondientes a las cuatro versiones del \textit{software}:
\begin{itemize}
  \item \texttt{contrast} -- versión original, secuencial
  \item \texttt{contrast-omp} -- versión paralelizada con OpenMP
  \item \texttt{contrast-mpi} -- versión paralelizada con MPI
  \item \texttt{contrast-mpi-omp} -- versión paralelizada de forma híbrida, con OpenMP y MPI
\end{itemize}



\subsection*{Ejecución}

Lo primero, es necesario convertir la imagen de entrada a los formatos PGM (\texttt{in.pgm}) y PPM (\texttt{in.ppm}). Ésto se puede realizar, por ejemplo, mediante el uso de la herramienta de línea de comandos Convert\footurl{https://imagemagick.org/script/convert.php}, y dejarla en el mismo directorio que el ejecutable, en nuestro caso, el directorio \texttt{build/}.

\noindent
Por ejemplo:
\begin{verbatim}
  convert highres.jpg build/in.pgm
  convert highres.jpg build/in.ppm
\end{verbatim}


Las distintas versiones pueden ser ejecutadas en un sistema de colas como Slurm\footurl{https://slurm.schedmd.com/documentation.html}, de la siguiente forma:
\begin{verbatim}
  srun -N <nodos> -n <procesos> <ejecutable>
\end{verbatim}

Especificando el número de nodos a usar y el número de procesos. Para la versión secuencial y la de OpenMP, estos números serán ambos $1$, ya que ambas ejecutan un único proceso.

Para la versión OpenMP y la híbrida, además se podrá especificar el número de \textit{threads} de ejecución mediante la variable de entorno \texttt{OMP\_NUM\_THREADS=<threads>}.

Al ejecutar el programa, se generarán tres ficheros de salida, \texttt{out.pgm}, \texttt{out\_hsl.ppm} y \texttt{out\_yuv.ppm}, correspondientes a los tres sub-procesos de ecualización.